

\section{1.実験の目的}
工業上、広く温度計測に用いられている熱電温度計の素線、すなわち熱電対特性を水の沸点とパラフィン系の融点(凝固点)
を基準点として、その校正曲線を求め、熱電温度計の原理の理解とその取扱いの習得を目的とする。

\section{2.熱電温度計の原理}
熱電温度計の温度センサー「熱電対」は、1821年にT.J.Seebeckによって発見された熱電流現象(ゼーベック効果)
にその基礎を置く。すなわち、相異なる2種類の金属線を用いて閉回路を作り、その二つの接合点(温接点、冷接点)
に温度差を与えるとその温度差にほぼ比例する起電力が生じて回路の電流が流れる。
この現象をゼーベック効果といい、その発生する起電力(熱起電力)は、熱電対の金属の種類をによって異なる。
この熱電対に生じる熱起電力を微小電圧計(mv計)で測定し、予め測定して定めておいた校正曲線(温度差と熱起電力の関係)によって、
電圧を温度に換算し温度計測を行うものを「熱電温度計」といい、まさにゼーベック効果を応用した計測法である。(図1参照)

熱電対の種類は、素線名では、銅・コンスタンタン(銅とニッケルの合金)、鉄・コンスタンタン、クロメル・コンスタンタン、
クロメル・アルメル、白金・白金ロジュウムなどがあり、JIS規格ではそれぞれをT、J、E、Kの記号で表示している。
それらの計測できる温度は、素線の特性によって決まり、ちなみに本実験で用いる銅・コンスタンタン(Tタイプ)は
-200℃ ~ 350℃となっている。

\section{3.使用機器}
\begin{itemize}
  \item 銅・コンスタンタン熱電対(φ0.2mm)
  \item 冷接点用電子冷却装置
  \item ミリボルト計
  \item ストップウォッチ
  \item パラフィン系試薬(ノルマル・オクタデカン、融点28℃)
  \item 三角フラスコ
  \item 卓上ヒーター(300W)
\end{itemize}



